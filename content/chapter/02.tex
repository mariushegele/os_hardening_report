
\chapter{Taktiken zur Härtung von Betriebssystemen}

% 1. Least Privilege Operation
%   1. Service Accounts
%   2. Seccomp-bpf
%   3. Setuid und Capabilities
%   4. MAC
%   5. Prozess Separierung
% 2. Sandboxing
%   1. Containerization
%   2. Virtualization

In diesem Teil soll auf mögliche Taktiken zur Härtung eines Linux-Betriebssystems gegen Angriffe unterschiedlicher Art eingegangen werden. Diese Auflistung ist nicht vollständig. Der Fokus liegt auf den Taktiken laufende Prozess mit geringst möglichen Privilegien zu versehen (Least Privilege) oder in einer sog.\ Sandbox abzuschirmen. Es werden weitere Techniken zur Härtung gegen Speicher-Korruption und andere Schwachstellen oder Angriffstaktitken beschrieben.

\section{Least Privilege Betrieb}

Saltzer und Schröder definieren 1975 das `Least-Privilege'-Prinzip als Betriebskonzept, in dem jedes Programm und jeder Nutzer unter den geringst möglichen Privilegien operieren, die notwendig sind, um die jeweilige Aufgabe zu erfüllen \cite{saltzer_1975}. Dies minimiert die Wahrscheinlichkeit eines Missbrauchs von Privilegien sowie die Zahl der Anwendungen die bei einem Missbrauch geprüft werden müssen. 
Dem privilegierten \texttt{root}-Nutzer stehen unter Linux alle Informationen sowie Kontrollmechanismen zur Verfügung. Kaum ein einzelner Prozess benötigt jedoch all diese. Ein Angreifer, der es schafft seine Privilegien über Schwachstellen bis auf dieses Level zu eskalieren, hat das System unter voller Kontrolle. Dies wird einfacher je mehr Prozesse und Nutzer Zugriffe haben, die sie nicht benötigen.

Das einfachste Beispiel ist das eines öffentlich-zugänglichen Services, wie eines Web Servers, der mit hohen Privilegien läuft. Welche dieser Privilegien benötigt er wirklich? Auf welche lässt sich dieser einschränken?
Ein sicheres Betriebskonzept beinhaltet eine Separierung von Privilegien mit der Anforderung an ein ``need-to-know'' im Bezug auf 
\begin{itemize}
  \item die nutzbaren System Calls,
  \item die Sicht auf laufende Prozesse
  \item und den Zugriff auf Informationen.
\end{itemize}


\subsection{Service Accounts}
\label{sec:service-acc} 

Linux Dateisysteme unterstützen das traditionelle Unix-Format für Zugriffsberechtigungen. Dies baut auf dem Prinzip `Discretionary Access Control' (DAC), bei dem die Zugriffsberechtigung zu einzelnen Dokumenten von der Identität des Nutzers abhängt. Dies unterscheidet sich insofern vom Least-Privilege-Prinzip, dass auch ein Nutzer höchsten Ranges nicht an jeder Aufgabe beteiligt ist und damit nicht jeden möglichen Zugriff benötigt.

Eine simple aber notwendige Maßnahme um Privilegien auf einem solchen Dateisystem zu separieren ist, dass jede Anwendung im Namen eines korrespondierenden Nutzers läuft. Nur dieser Nutzer hat Zugriff auf die App-spezifischen Daten und vor allem keine Einsicht in die Informationen anderer Anwendungen oder des Systems. Dies ist der grundlegende Baustein des Android Sandbox Modells \cite{android-sandbox}. Jeder App wird ein eigenes Verzeichnis und eine eigene UID zugewiesen.

Angenommen es existiert eine einfache Browser Anwendung, die über eine Sicherheitslücke Angreifern Arbitrary Code Execution unter den Privilegien des Prozesses ermöglicht. Ein Angreifer hätte potenziell Interesse, die Inhalte anderer Anwendungen, wie des Kalenders zu erspähen.
Statt den Browser und den Kalender und deren Daten unter derselben UID zu halten, sollte beide ihre eigenen ``Service-Accounts'' haben. Unter Linux sind bereits bestimmte UID-Bereiche für solche Nutzer reserviert. Wird ein Nutzer mit \texttt{adduser -{}-system} erstellt, so ist es nicht möglich sich als dieser Nutzer anzumelden (z.B. über SSH). Die Zugriffsberechtigung (\texttt{umask}) der Anwendungsverzeichnisse sollten auf einzig Zugriff durch den Inhaber (Service-Account) beschränkt werden.

\begin{lstlisting}[language=bash]
$ sudo -u browser cat calendar/data.txt
Today I have a doctors appointment at 4 pm to check up on my stomach aches.

$ sudo adduser --system --home /apps/calendar calendar
$ sudo chmod 700 /apps/calendar

$ sudo -u browser cat calendar/data.txt
cat: /apps/calendar/data.txt: Permission denied
\end{lstlisting}

Um sicherzustellen, dass alle Daten in den jeweiligen Applikations-Verzeichnissen unter derselben DAC-Policy (700) erstellt werden, sollten Default-Permissions über \texttt{umask} und Access Control Listen (ACL) gesetzt werden. 

\begin{lstlisting}[language=bash]
$ setfacl -d -m g::--- /apps # set group to none default
$ setfacl -d -m o::--- /apps # set other to none default
\end{lstlisting}

Damit sich dies auch auf über \texttt{adduser -{}-system -{}-home=/apps/<app>} automatisch erstellte Nutzer-Verzeichnisse bezieht, sollte der \texttt{UMASK} Parameter unter \texttt{/etc/login.defs} auf 700 konfiguriert werden. Damit Prozesse nicht eigenständig die Inhaber ihrer Dateien ändern können (\texttt{chmod}), sollten ihnen die beiden System Calls \texttt{chown} und \texttt{chmod} über \texttt{seccomp}-Filter verboten werden. 

\subsection{Seccomp-Filter}
\label{sec:seccomp}

\texttt{Seccomp} wurde entwickelt, um das Ausführen von unvertraulichem Code, wie zum Beispiel im Grid Computing oder in Browsern, zu ermöglichen. Es erlaubt es Black- oder Whitelisten für System Calls zu definieren. So wird die Schnittstelle zwischen Userland und privilegiertem Kernel minimiert.
Im `strict'-Modus werden nur 4 rudimentäre System Calls erlaubt: \texttt{read, write, \_exit, rt\_sigreturn} \cite{man-seccomp}. Der \texttt{fopen} Call ist nicht verfügbar. Es können also Dateien nur über bereits geöffnete File Deskriptoren bearbeitet oder gelesen werden. Dieser Grad der Einschränkung ist für viele Anwendungsfälle zu hoch.
Die Filter können nicht auf Prozesse angewendet werden, die ohne \texttt{CAP\_SYS\_ADMIN}-Privilegien laufen und neue Privilegien erteilen\footnote{zum Beisipel über die Ausführung über \texttt{execve} eines Setuid Programms} \cite{man-seccomp, man-prctl}.

Seccomp bietet unterschiedliche Möglichkeiten auf eine Überschreitung der definierten Policy zu reagieren. Der verbotene System Call kann von der Ausführung gestoppt werden und der aufrufende Prozess (unter anderen Optionen) beendet (\texttt{kill}) oder fortgeführt werden. Um Profile für existierende Anwendungen anzulegen bietet sich der Modus an, Überschreitungen zu loggen und trotzdem auszuführen. Eine Anwendung die versucht Zugriffsrechte für ihre Dateien zu ändern, sollte zum Beispiel beendet
werden:

\begin{lstlisting}[language=c]
scmp_filter_ctx ctx;
ctx = seccomp_init(SCMP_ACT_ALLOW);

// Black List:
ret |= seccomp_rule_add(ctx, SCMP_ACT_KILL, SCMP_SYS(fchmodat), 0); 
\end{lstlisting}

Um im Einklang mit dem Least-Privilege-Betriebs zu bleiben, sollte statt einem Blacklist-Ansatz\footnote{\texttt{SCMP\_ACT\_ALLOW} als Default im \texttt{init} mit Verboten einzelner Calls über \texttt{SCMP\_ACT\_KILL}} ein Whitelist-Ansatz verfolgt werden, bei dem nur die tatsächlichen notwendigen Rechte erteilt werden. Für viele Anwendungen genügt die Freiheit Dateien zu öffnen, zu lesen und zu schreiben\footnote{in begrenzter Anzahl und Größe um DoS Attacken zu vermeiden}.
Alternativ können Listen wie Docker's Standardprofil bei der Identifizierung kritischer System Calls wie \texttt{ptrace}, \texttt{create\_module} und \texttt{reboot} helfen \cite{docker-seccomp}. 

Mit \texttt{seccomp} gibt es auch die Möglichkeit, System Calls in dem Raum der validen Argumente einzuschränken. So kann zum Beispiel die Nutzung von \texttt{chmod} feiner eingeschränkt werden, indem eine Änderung der Rechte nur auf geringeres als \texttt{rwx} für den Inhaber erlaubt wird. Argumente an System Calls werden über Register übergeben \cite{man-syscall}. Sind Argumente numerische Werte, so kann \texttt{seccomp} die entsprechenden Register prüfen,
doch sind sie Pointer so kann es nur diesen prüfen jedoch nicht den referenzierten Wert. Deswegen können Strings, wie zum Beispiel Dateipfade als Argumente an \texttt{open}, nicht gefiltert werden. Für solche Einschränkungen benötigt es andere Sicherheits-Vorkehrungen.

\subsection{Prozess Separierung}

Ebenso wie die Verzeichnisse anderer Anwendungen, sollten auch die laufenden Prozesse anderer Anwendungen nicht sichtbar sein. Standardmäßig sind die einem Prozess korrespondierenden Verzeichnisse unter \texttt{/proc/<pid>} für jeden sichtbar. Diese enthalten sensitive lesbare Informationen wie \texttt{cmdline}, was die Kommandzeile mit Argumenten anzeigt, mit der der Prozess gestartet wurde oder \texttt{status}, was nützliche Informationen wie den Zustand des Prozesses, dessen
UID, dessen Parent ID, dessen Speichergröße oder dessen \texttt{seccomp} Modus anzeigt. Ein weitere sensitive Information ist das Speicher Layout \texttt{maps}, vor allem in Systemen, die Address Space Layout Randomization benutzen (siehe Abschnitt \ref{sec:aslr}).
Angreifer nutzen diese Informationen zur Process Injection und damit zur Privilege Escalation und Defense Evasion \cite{attack-process-injection}.
Um diese Weltoffenheit vertraulicher Informationen zu reduzieren, kann \texttt{/proc} mit der \texttt{hidepid=2} Flag gemountet werden. Dies bewirkt, dass gennante Verzeichnisse nicht mehr sichtbar sind und damit die PID's der laufenden Prozesse nicht mehr in Form eines Verzeichnisses und nicht mehr unter \texttt{ps aux} auftreten. 

\begin{lstlisting}[language=bash]
$ sudo -u calendar sleep 3600 &
[1] 3508

$ sudo -u browser cat /proc/3508/cmdline | strings -1
sleep
3600

$ sudo mount -o remount,rw,hidepid=2 /proc

$ sudo -u browser cat /proc/3508/cmdline | strings -1
cat: /proc/3508/cmdline: No such file or directory
\end{lstlisting}

Prozesse können natürlich weiterhin über Abtasten (z.B. über \texttt{kill}) erkannt werden. Weitere Sicherheitsvorkehrungen beim Mounten von \texttt{/proc} können die Optionen \texttt{nodev}, \texttt{noexec} und \texttt{nosuid} sein. Diese verhindern die Verwendung von Special Devices und damit direkten Zugriff auf Hardware, das Ausführen von Binaries in diesem Filesystem bzw.\ das Respektieren von Setuid-Bits oder Capabilities. Alle diese Funktionen sollten unter \texttt{/proc} keine Anwendung finden und können damit deaktiviert werden.

Die Nutzung von \texttt{ptrace} sollte für Service-Accounts unmöglich gemacht werden. Es erlaubt, Prozesse in der Laufzeit zu verfolgen und zu modifizieren \cite{man-ptrace}. Dies wird zum Beispiel für Debugging genutzt, kann aber auch einem Angreifer als mächtiges Tool zur Privilege Escalation dienen \cite{attack-process-injection}.
Außerdem eröffnet es unter Linux Kernel < 4.8 die Möglichkeit über \texttt{SECCOMP\_RET\_TRACE} sämtliche \texttt{seccomp}-Filter zu umgehen \cite{ptrace-seccomp-bypass}.
\texttt{Ptrace} kann über \texttt{sysctl}-Konfiguration des Yama-LSM darauf beschränkt werden dass niemand, nur Admins oder nur Parent-Prozesse (\texttt{gdb <bin>}) Einblick den laufenden Prozess erhalten. \texttt{Ptrace} nutzt Core Dumps, um Prozesse zu analysieren. Diese entsprechen Momentaufnahmen des Zustands des Speichers, der Register, des Stacks und anderer Eigenschaften des Prozesses. Ein Prozess kann über \texttt{prctl} als nicht `dump'-bar konfiguriert werden \cite{man-proc}. In diesem
Falle lässt er sich nicht tracen \cite{man-ptrace}.

% TODO Test via filter.c + 
% ptrace(PTRACE_ATTACH, <pid>, NULL, NULL);

\subsection{Setuid und Capabilities}

Beim Ausführen einer Datei erhält der resultierende Prozess im Normalfall die Nutzerrechte des Aufrufenden. Eine Ausnahme zu dieser Regel herrscht, wenn die ausgeführte Datei mit einem Setuid-Bit versehen ist. Dies bewirkt, dass der resultierende Prozess stattdessen die Zugriffsrechte des Dateiinhabers erhält \cite{man-chmod}. Ist der Dateinhaber dabei \texttt{root}, so stellen ausnutzbare Schwachstellen in diesen Programmen einen Schwachpunkt dar, der Privilege Escalation ermöglicht. Dasselbe gilt analog für Setgid
Bits und Nutzergruppen. Diese Eigenschaft wird des Weiteren benutzt, um erlangte erhöhte Privilegien auf einem System über Neustarts hinweg zu persistieren \cite{attack-setuid}. Aus den genannten Gründen, sollte die Zahl der existierenden Setuid Anwendungen auf ein Minimum reduziert werden. Dies kann erreicht werden, indem einer Anwendung bestimmte Rechte (Capabilities) zu privilegierten Operationen eingeräumt werden, die normalerweise nur \texttt{root} in der Lage ist
auszuführen. 

\begin{lstlisting}[language=c,label=lst:timetravel]
dest = now - 60;
ret = stime(&dest);
if(ret != 0)
{
    printf("Failed to travel in time.\n");
    return -1;
}
now = time(NULL);
printf("We are in the past!!1!\n");
print_time(&now);

/* mocking some vulnerability */
setuid(0); // privilege escalation
system("cat ~/safe/precious_data.txt && echo 'you have been pwned'");
\end{lstlisting}

Die Gefahr der Nutzung von Setuid Bits lässt sich an einem Zeitreise-Beispiel demonstrieren. Hierzu wird der \texttt{stime} System Call verwendet, der die \texttt{CAP\_SYS\_TIME} Capability erfordert \cite{man-capabilities}. Nun bieten sich zwei Möglichkeiten diese der Anwendung bei der Ausführung zu erteilen: die \texttt{root}-Inhaberschaft verbunden mit einem Setuid Bit oder das einfache Erteilen der Capability.


\begin{lstlisting}[language=bash]
$ sudo chown root:root time_travel_suid
$ sudo chmod u+s time_travel_suid 
# versus
$ sudo setcap CAP_SYS_TIME=eip time_travel_cap

$ ll time_travel*
-rwxr-xr-x 1 root  root  17080 Apr 29 09:08 time_travel*
-rw-rw-r-- 1 otter otter   727 Apr 29 09:09 time_travel.c
-rwxrwxr-x 1 otter otter 17080 Apr 29 09:08 time_travel_cap*
-rwsrwxr-x 1 root  root  17080 Apr 29 09:08 time_travel_suid*
\end{lstlisting}

Das \texttt{s} auf dem Setuid-Binary indiziert das gesetzte Setuid-Bit. Eine Ausführung des SUID-Binary führt zu Privilege Escalation nach `Ausnutzen' der gemockten Vulnerability und damit zum Verlust der Vertraulichkeit auf einer sonst unlesbaren Datei.

Die Setuid Funktionalität kann per Block Device mithilfe von \texttt{mount -o nosuid} deaktiviert werden. Doch dies verbietet auch das Zuweisen von Capabilities. Dies kann also nicht für alle Dateisysteme eine Lösung sein. Stattdessen sollte
die Existenz von Setuid-Binaries beobachtet und über Capabilities minimiert werden.

\subsection{Mandatory Access Control}

Capabilities sind ein Beispiel für die Umsetzung eines alternativen Ansatzes zur Zugriffsberechtigung -- `Mandatory Access Control' (MAC). Im Allgemeinen wird dabei ein Regelwerk definiert, das bestimmt, ob ein Nutzer oder ein Prozess eine Operation auf einer bestimmten Ressource ausführen darf. Dabei werden nur die explizit als erlaubt definierten Operationen auch durchgeführt. Es existieren Linux Security Module zur System-weiten Umsetzung von MAC. Die zwei bekanntesten davon
sind AppArmor und SELinux. Diese erlauben die Definition von Security-Profilen für jede ausführbare Datei. Diese haben neben den zugewiesenen Capablities potenziell auch Berechtigungen oder Verbote zum Zugriff auf Verzeichnisse.

Für Legacy-Setuid-Anwendungen mag die Definition aller notwendigen Capabilities aufwendig sein. In solchen Fällen hilft AppArmor bei der Generierung eines Profils. Wird der Kommandozeilenbefehl \texttt{aa-genprof} auf einem Executable ausgeführt, so generiert
AppArmor ein leeres Profil, konfiguriert das definierte Programm im `Complain'-Mode und berichtet bei Überschreitungen der (dato undefinierten) Rechte. So kann für jede der auffallenden Überschreitungen entschieden werden, ob das Recht erlaubt oder verboten werden sollte. So könnte im demonstrierten Fall die Capability \texttt{CAP\_SYS\_TIME} erlaubt, aber der Zugriff auf \texttt{/safe} verboten werden:

\begin{lstlisting}
#include <tunables/global>

/home/otter/cap_pen_test/time_travel_cap {
  #include <abstractions/base>

  capability sys_time,
  deny /safe rwx,

  /home/otter/cap_pen_test/time_travel_cap mr,
}
\end{lstlisting}



\section{Sandboxing}

Ein Drive-by Compromise ist eine Attacke, bei der der Prozess des Browsers auf dem Rechner des Nutzers über den Besuch einer Webseite und das Ausnutzen einer Schwachstelle übernommen wird. Nebem dem Browser sind Office-Anwendungen und bekannte Third-Party-Anwendungen ein beliebeter Angriffsvektor. Da diese Anwendungen weit verbreitet und Nutzer mit deren Handhabung vertraut sind, bieten diese ein gutes Einstiegsfenster für Angreifer, die über Spearfishing Links oder Dokumente an ihre Opfer
versenden. Das schlichte Öffnen dieser kann dabei bereits genügen, um dem Angreifer Arbitrary Code Execution zu verschaffen. Die genannten Anwendungen sind groß, komplex und werden kontinuerlich mit neuen Featuren weiterentwickelt. Aus diesen Gründen werden sie vielleicht sogar nie fehlerfrei sein. Die Verwendung dieser ist für viele Unternehmen jedoch geschäftskritisch. Der einzige Weg trotz der Verwendung solcher Anwendungen, die Sicherheit zu wahren ist es, diese bestmöglich
`abzuschirmen', damit die gennanten Ausbrüche möglichst effektlos bleiben.

Es existieren bereits eine Menge an Ansätzen zum Sandboxing von Anwendungen in einem Linux OS (siehe Tabelle \ref{tab:sandbox-apps}).
Es gibt Betriebbsysteme, die auf dem Prinzip des Sandboxing basieren. QubesOS kapselt so viel wie möglich von einander ab um die Folgen eines Compromise einer Komponente maximal zu reduzieren \cite{qubes}.
Die bereits beschriebene Separierung und Minimierung von Privilegien ist ein wichtiger Bestandteil dieses `Sandboxing'-Ansatzes. Ist ein Angreifer in der Lage, aus dem abgeschirmten Bereich auszubrechen sollte er dennoch so wenig Privilegien wie möglich haben.

\begin{table}[]
\centering
\resizebox{0.95\textwidth}{!}{%
\begin{tabular}{|l|c|c|c|c|c|c|c|c|}
\hline
\textbf{Sandbox}                               & \textbf{Seccomp}  & \textbf{Namespaces}   & \textbf{HW-Virt.} & \textbf{MAC}      & \textbf{Capabilities}  & \textbf{chroot}    & \textbf{ulimit}   & \textbf{cgroups}  \\ \hline
firejail           \cite{firejail}             & \checkb            & \checkb                & \nocheckb          & \checkb        & \checkb                & \checkb            & \nocheckb          & \nocheckb          \\ \hline
bubblewrap         \cite{bubblewrap}           & \checkb            & \checkb                & \nocheckb          & \nocheckb      & \checkb                & \checkb            & \nocheckb          & \nocheckb          \\ \hline
LXC                \cite{lxc}                  & \checkb            & \checkb                & \nocheckb          & \checkb        & \checkb                & \nocheckb          & \checkb            & \checkb            \\ \hline
Docker             \cite{docker}               & \checkb            & \checkb                & \nocheckb          & \checkb        & \checkb                & \nocheckb          & \checkb            & \checkb            \\ \hline
SELinux Sandbox    \cite{selinux-sandbox}      & \nocheckb          & \checkb                & \nocheckb          & \checkb        & \checkb                & \nocheckb          & \nocheckb          & \checkb            \\ \hline
libvirt-sandbox    \cite{libvirt-sandbox}      & \nocheckb          & \checkb                & \checkb            & \checkb        & \nocheckb              & \nocheckb          & \nocheckb          & \nocheckb          \\ \hline
Cuckoo             \cite{cuckoo}               & \nocheckb          & \nocheckb              & \checkb            & \nocheckb      & \nocheckb              & \nocheckb          & \nocheckb          & \nocheckb          \\ \hline
QubesOS            \cite{qubes}                & \nocheckb          & \nocheckb              & \checkb            & \nocheckb      & \nocheckb              & \nocheckb          & \nocheckb          & \nocheckb          \\ \hline
mbox               \cite{mbox}                 & \checkb            & \nocheckb              & \nocheckb          & \nocheckb      & \nocheckb              & \nocheckb          & \nocheckb          & \nocheckb          \\ \hline
\end{tabular}%
}
\caption{Sandbox Anwendungen für Linux und deren eingesetzte Komponenten. Diese Auflistung ist weder in den aufgeführten Anwendungen, noch in deren Komponenten erschöpfend.}
\label{tab:sandbox-apps}
\end{table}
% TODO runC
% TODO Chromium, Firefox
% TODO systemd-nspawn
% TODO Solaris Zones


Hier sollen zwei Sandboxing-Ansätze näher vorgestellt werden: 

\begin{itemize}
    \item Hardware Virtualisierung
    \item OS-Level Virtualisierung (Namespaces, Seccomp, Capabilities und cgroups)
\end{itemize}

\subsection{Hardware Virtualisierung}

Bei der Hardware Virtualisierung werden virtuelle Maschinen erstellt. Diese entsprechen in der Benutzung der eines separaten Computers mit dedizierten Hardware Ressourcen sowie einem Betriebssystem. Die Wahl des Betriebssystems ist komplett unabhängig vom Host System. Der Host fungiert in diesen Fällen als Virtual Machine Monitor (VMM; Hypervisor) \cite{uhlig-intel}. Er verwaltet den Zugriff auf Ressourcen und bietet dafür eine Virtual Operating Platform als Schnittstelle.
Hypervisor `ersten' Typs werden auch Bare-Metal-Hypervisor genannt und laufen direkt auf der Hardware des Hosts. Hypervisor `zweiten' Grades hingegen sind Anwendungen in einem Betriebssystem. Es gibt Hypervisor, die hybride Form dieser beiden Typen darstellen, indem sie den Kernel über Kernel-Module zu einem Hypervisor erweitern.
Um ein gehärtetes Betriebssystemen zu erreichen sollte also entweder ein Typ-2-Hypervisor unter hoher Absicherung verwendet werden (z.B.\ \texttt{libvirt-sandbox} \cite{libvirt-sandbox}), oder ein Betriebssystem auf der Basis eines Typ-1-Hypervisor definiert werden (z.B.\ QubesOS \cite{qubes}). 

Es wird des Weiteren zwischen voller Virtualisierung und Paravirtualisierung unterschieden \cite{xen}. Bei der vollen Virtualisierung müssen Betriebssysteme nicht angepasst werden, da hier die Hardware simuliert wird. Paravirtualisierung bot vor der Entwicklung von Hardware-Virtualisierung eine Leistungsverbesserung, da hier nicht jede Instruktion auf der simulierten Hardware, sondern bestimmte Aufgaben auf der tatsächlichen Hardware ausgeführt werden. Dies löste manche Probleme, die
dadurch aufkamen, dass Betriebssysteme nicht mehr auf dem Hardware-Privilege-Ring liefen, für die sie ursprünglich geschrieben wurden \cite{uhlig-intel}. Erst durch den Fortschritt der Hardware-Virtualisierung (Intel VT-x, AMD-V) wird dieses Problem jedoch richtig addressiert. Die CPU Hersteller haben dafür zwei unterschiedliche Modi eingeführt, zwischen denen bei der Virtualisierung gewechselt wird. Der Hypervisor läuft im `Root'-Modus auf allen vier Privilegien-Ringen. Das Gast-OS
und deren Anwendungen laufen im `Non-Root'-Modus wieder auf den vorgesehenen Ringen 0 bzw. 3 \cite{uhlig-intel}. Um Instruktionen auszuführen und Ressourcen anzufordern stellt das Gast-OS einen sog.\ ``Hypercall'' an den Hypervisor. Dieses Vorgehen erfordert wenige bis keine Änderungen aufseiten des Betriebssystems. 

Die Vorteile von Virtualisierung sind vielseitig. Es ermöglicht homogenere und dadurch leichter zu wartende Platformen durch Konsolidierung. Es vereinfacht die Migration und den Betrieb\footnote{insb. Load Balancing und Fehlervorhersage} von Systemen \cite{uhlig-intel}. Der Aspekt, der für die Härtung von Betriebssystemen jedoch relevant ist, ist der der Isolation. Durch das `Einsperren' von unvertraulichen Anwendungen kann ein Eindringen verhindert werden, ein fehler-tolerantes System
entwickelt werden \cite{terra} oder ein Angriff durch Logging nachvollzogen werden \cite{revirt}.

Der Einsatz von Hardware Virtualisierung zur Härtung kann wie beschrieben helfen, jedoch auch neue Schwachstellen einführen.
Ein Ausbrauch aus einer Gastmaschine ermöglicht lokale Angriffe auf andere Gastmaschinen, die ohne Virtualisierung über physisches Netzwerk getrennt wären \cite{svirt}. Die Verlässlichkeit des Hypervisors ist deshalb kritisch.
Werden mehrere virtuelle Maschinen auf einem Desktop verwendet, so stellen vor allem ein geteiltes Clipboard und Keylogging potenzielle Gefahren dar. 
Es sollten illegale Änderungen am Speicher verhindert werden und nur authorisiert Code ausgeführt werden. Das Verhalten des Gast OS kann über Kontrollflussanalyse und definierten Policies beobachtet und dadurch Eingriffe verhindert werden \cite{christo}.

\subsubsection{QubesOS}

QubesOS ist der Versuch ein `Security-first' Betriebssystem über den Ansatz der Isolation von so vielen Komponenten wie möglich zu erreichen \cite{qubes}. Die Kompromittierung einer einzelnen soll nie zum Verlust der Integrität einer anderen führen.
Es baut auf den schmalen quelloffenen Xen Typ-1-Hypervisor, den es über $W \oplus X$ und ASLR absichert (siehe Abschnitt \ref{sec:mem-sec}).

Es wird versucht die größe der Trusted Computing Base (TCB) zu minimieren. Die Netzwerk- und Speicherdomäne werden dafür in separate virtuelle Maschinen verlagert. Des Weiteren lassen sich Anwendungen nach bestimmten realistischen
Bereichen in separaten VMs betreiben. Die Webseite für das Online-Banking muss nicht im selben Browser besucht werden, der auch für soziale Netzwerk benutzt wird. Pro Bereich lässt sich der Netzwerk Traffic auf bestimmte Ziele beschränken. Geschäftsbezogene Dokumente sollten von privaten separiert sein.

Die Nutzbarkeit des Multi-VM-Ansatzes ist von einer üblichen GUI, einem geteilten Clipboard, zentralen Updates und vertretbarem Speicheraufwand abhängig. Zentrale Updates werden über ein geteiltes signiertes read-only Block Device für \texttt{/usr, /bin} und ähnliche Verzeichnisse implementiert. Die Einfüge-Operation für das geteilte Clipboard wird dadurch abgesichert, dass sie nur von der TCB initialisiert werden kann. Das Teilen von Dateien über zwei Anwendungs VMs wird über
Verschlüsselung und den vertraulichen Xen Store gesichert. 

Die Netzwerk VM beinhaltet Treiber und Protokolle, die mit hohen Privilegien kommunizieren. Sie bietet über Xen sichere virtuelle Netzwerk Interfaces für jede Anwendungs VM. Die Storage VM isoliert Attacken über entfernbare Geräte. Damit die Vertraulichkeit und Integrität gewahrt wird, kann es das geteilte Dateisystem nicht ändern und die Dateien einer Anwendungs-VM weder lesen noch schreiben.


\subsection{OS-Level Virtualisierung}

Virtualisierung auf OS-Ebene ist nicht mit der Verwendung eines Typ-2-Hypervisors zu verwechseln. Stattdessen werden Funktionen des Kernels benutzt, um ein vergleichbares Niveau an Isolation herzustellen.
Das Ziel ist es auf Anwendungsebene eine Umgebung schaffen, die so nah wie möglich an der einer virtuellen Maschine ist, ohne den Overhead eines separaten Kernels und der Hardware Emulation zu haben \cite{lxc}.

Chromium und Firefox setzen in ihrem Sandboxing-Ansatz primär auf Restriktion von System Calls über Seccomp, Ressourcenbeschränkung über cgroups sowie Namespaces \cite{firefox-sandbox, chromium-sandbox}. 
Vor allem bei der Auswahl an Tools zur Virtualisierung auf Betriebssystemebene, sticht die Ähnlichkeit in der Nutzung dieser Linux Funktionen heraus. Zuerst OpenVZ \cite{openvz}, dann LXC
\cite{lxc}, später Docker \cite{docker} und letztendlich runc \cite{runc} bauen auf Kernel namespaces, AppArmor oder SELinux, Seccomp, Chroot `jails', Capabilities und cgroups. 
Tabelle \ref{tab:sandbox-apps} zeigt wie stark diese Funktionalitäten auch bei anderen Sandboxing Tools Verwendung finden.

Auf Seccomp, Capabilities und AppArmor wurde bereits eingegangen. Diese werden
benutzt, um die Wahrscheinlichkeit einer Privilege Escalation im Falle eines Container Escapes zu reduzieren und minimieren dafür die Oberfläche des Kernels und des File Systems. In diesem Abschnitt soll daher näher auf chroot, Namespaces, sowie cgroups eingegangen werden.

Linux Namespaces sind als leichtgewichtige Variante zu Virtualisierung ohne Hypervisor zu sehen. Sie funktionieren ähnlich zur Funktionalität von \texttt{chroot}. Dieses kann benutzt werden, um das root Verzeichnis für den aktuellen Prozess zu ändern. Für diesen Prozess sind im Nachhinein Dateien eines Verzeichnisses, das über dem spezifizierten liegt, unsichtbar. Es ist jedoch nicht als Sandbox-Mechanismus gedacht. Aus einer chroot Sandbox kann
durch simples Wechseln und Bewegen von Verzeichnissen oder ein zweites \texttt{chroot} ausgebrochen werden \cite{man-chroot, second-chroot}. Alleinstehend genügt es also nicht für einen effektiven Sandbox-Mechanismus. Wenn es verwendet wird, sollte darauf geachtet werden das Privilegien sofort nach dem \texttt{chroot} reduziert werden und dass das Verzeichnis keine Schwachstellen enthält, die eine Privilege Escalation ermöglichen (z.B. setuid-root Binaries) \cite{second-chroot}.
Bubblewrap verbindet die Nutzung von chroot mit der Prozess Eigenschaft \texttt{PR\_SET\_NO\_NEW\_PRIVS} \cite{bubblewrap}.

Auf dieselbe Art und Weise isoliert ein Linux Namespace zum Beispiel Mount Punkte. Änderungen an den Mount Punkten eines Namespace
werden nicht auf anderen repliziert \cite{man-mount-ns}.
Es gibt unterschiedliche Arten von Namespaces, die allen nach demselben Prinzip der Isolation von Ressourcen oder Informationen funktionieren.

\begin{table}[H]
\label{tab:ns}
\centering
\resizebox{0.9\textwidth}{!}{%
\begin{tabular}{lll}
\textbf{Namespace} & \textbf{Isoliert}                  & \textbf{Hierarchisch} \\ \hline
User               & User und Group IDs                 & Ja    \\
PID                & Prozess IDs                        & Ja    \\
Cgroup             & Cgroup root Verzeichnis            & Ja    \\
Mount              & Mount Punkte                       & Nein  \\
Network            & Network Devices, Stacks, Ports     & Nein  \\
UTS                & Hostname und NIS domain name       & Nein  \\
IPC                & System V IPC, POSIX Message queues & Nein
\end{tabular}%
}
\caption{Aus \cite{man-ns-7}}
\end{table}

Beim Erstellen eines hierarchischen Namespaces werden die Einstellungen des aktuellen Prozesses oder Nutzers an den `Child'-Namespace vererbt. Änderungen an diesen Kindern werden nicht in der Hierarchie nach oben propagiert \cite{man-ns-7}. 

Durch \textbf{PID Namespaces} sind auf dem Host laufende Prozesse im Container nicht weiter sichtbar. Auch wenn der init Prozess im Container PID 1 hat und damit Wurzel des Prozessbaumes ist, ist er für den Host nur ein weiterer Prozess mit einer PID ungleich 1. Verschiedene Anwendungen können ihre Prozesse frei segmentieren und müssen dabei nicht auf die Existenz von Prozessen anderer
Anwendungen achten. Ähnlich zur \texttt{hidepid} mount-Option kann damit die Sicht auf fremde Prozesse reduziert und damit die Discovery-Phase erschwert werden.

\begin{lstlisting}[language=sh,caption={Isolation von Prozessen über PID namespaces}]
$ ps axo user,pid,command
USER       PID  COMMAND
root         1  /sbin/init maybe-ubiquity
root         2  [kthreadd]
...

$ unshare --fork --pid --mount-proc bash
$ ps axo user,pid,command
USER       PID  COMMAND
root         1  bash
root         8  ps aux
\end{lstlisting}

Ein Linux OS teilt sich Netzwerk Interfaces, Routing-Tabellen und Routing-Regeln. Mit \textbf{Netzwerk Namespaces} können separate Tabellen und Regeln definiert werden. Die beschränkte Sicht auf existierende Netzwerk-Interfaces reduziert die Wahrscheinlichkeit, dass ein Angreifender potenzielle Vektoren entdeckt.

\begin{lstlisting}[language=sh,basicstyle=\footnotesize\ttfamily,caption={Isolation der aktiven Internetverbindungen durch einen Netzwerk Namespace}]
$ netstat -ant
Active Internet connections (servers and established)
Proto Recv-Q Send-Q Local Address   Foreign Address  State
tcp        0      0 127.0.0.53:53   0.0.0.0:*        LISTEN
tcp        0      0 0.0.0.0:22      0.0.0.0:*        LISTEN
tcp        0     36 10.0.2.15:22    10.0.2.2:51046   ESTABLISHED
tcp6       0      0 :::22           :::*             LISTEN

$ unshare --net bash
$ netstat -ant
Active Internet connections (servers and established)
Proto Recv-Q Send-Q Local Address   Foreign Address  State
\end{lstlisting}

\textbf{Mount namespaces} isolieren die Sicht auf Mount Punkte. Es zeigte sich, dass Isolation in manchen Erfahrungen sogar zu hoch war, weswegen geteilte Unterbäume (shared subtrees) entwickelt wurden. Über diese können Mount Events über Namespaces hinweg in sogenannte slave mount Punkte propagiert werden. Mount Namespaces sind an Owner User Namespaces gebunden. Unprivilegierte Mount Namespaces sind Namespaces, bei denen sich der Owner NS vom Owner NS seines Parent Mount Namespaces
unterscheidet. In diesen Mount NS sind alle Mount Punkte Slave Mounts \cite{man-mount-ns}. Dieser Mechanismus eignet sich gut zur Reduzierung der Container-Privilegien. In einem Container sollten in Normalfall keine weiteren Mount Namespaces notwendig sein.

\textbf{Control groups} (cgroups) ermöglichen es, die nutzbaren Ressourcen (CPU, Speicher etc.) für einzelne Prozessgruppen zu limitieren \cite{man-cgroups}. Dies ist eine wichtige Maßnahme, um Denial of Service Angriffe zu verhindern \cite{lxc-sec}. Ohne diese Limits könnten über eine Vielzahl an Requests an einen Web Server die Ressourcen des Hosts aufgebraucht werden, was zum Absturz potenziell sehr vieler Services führen würde.

\begin{lstlisting}[language=sh,caption={Ressourcenlimitierung einer DoS-Attacke (durch \texttt{yes} mimiert) mithilfe von cgroups}]
$ yes > /dev/null &
$ ps o user,pid,pcpu,command
USER       PID %CPU COMMAND
root      5546  149 yes

$ mkdir /sys/fs/cgroup/cpu/cg1
$ echo 100000 > /sys/fs/cgroup/cpu/cg1/cpu.cfs_quota_us
$ echo 1000000 > /sys/fs/cgroup/cpu/cg1/cpu.cfs_quota_us
$ echo 5546 > /sys/fs/cgroup/cpu/cg1/cgroup.procs

$ ps o user,pid,pcpu,command
USER       PID %CPU COMMAND
root      5546  9.8 yes
\end{lstlisting}

Cgroup Namespaces isolieren die Sicht auf die cgroups eines Prozesses. Diese scheinen jedoch nicht wichtig für die Funktionsweise von Containern zu sein.

Der wohl kritischste Punkt an der Container-Sicherheit ist die Verwendung von \textbf{User Namespaces}. Jede UID in einem Namespace ist auf eine UID im darüberliegenden Namespace gemappt. Dies eliminiert den Verwaltungsaufwand, mehrere Services und deren verwendete UIDs miteinander zu vereinbaren.

Jeder Namespace hat standardmäßig die normale UID Range von 0 bis 65535. In einem User Namespace können Prozesse also die effektive UID 0 und damit aus Kernelsicht \texttt{root} Rechte haben, obwohl sie von unprivilegierten Nutzern (UID $\neq 0$) abstammen. Ein unprivilegierter Nutzer kann einen Namespace erstellen. In diesem hat er das volle Set an Kernel Capabilities. Der Effekt dieser Capabilities bezieht jedoch nur auf die Resourcen, über die der User Namespace verfügt. Capabilities, die
nicht von solchen Ressourcen abhängen, wie zum Beispiel die Systemzeit, lassen sich nicht auf diese Weise erlangen \cite{man-user-ns}.
Dieser Aspekt kann zur Privilege Escalation ausgenutzt werden, sobald es aufseiten des Kernels eine Schwachstelle gibt, die dem Umstand der erhöten Privilegien entspringt. Die erhöhten Capabilities verstärken Probleme durch Schwachstellen im Kernel Code, zum Beispiel durch das Erlangen von \texttt{CAP\_NET\_ADMIN}, was unprivilegierten Nutzern das Programmieren von iptables erlaubt. \cite{lwn-controlling-access}. Kernel-Entwickler mögen weniger auf Sicherheitsaspekte bedacht sein, wenn ein
User \texttt{root} Rechte hat \cite{kerrisk-anatomy}. Dies zeigte
sich in Form mehrerer Schwachstellen \cite{cve-2013-1858, cve-2014-5206, cve-2014-5207, cve-2016-3135, cve-2017-1000111}. Das ist der Grund, weshalb manche Distributionen User Namespaces nur für privilegierte Nutzer unterstützen. Es gibt eine \texttt{sysctl} Konfiguration \texttt{kernel.unprivileged\_userns\_clone}, die es auch unprivilegierten Nutzern erlaubt, Namespaces zu erstellen. Diese sollte deaktiviert sein, um die beschriebene Gefahr der Privilege Escalation
zu vermeiden. Auf Ubuntu Server 19.10 ist sie per Default aktiviert.

\begin{lstlisting}[language=sh,caption={Eliminieren von root-Setuid Vulnerabilities mithilfe von User Namespaces}]
$ ll /usr/bin/ping
-rwsr-xr-x 1 root root 72776 Oct  5  2019 /usr/bin/ping*

$ ping google.com
PING google.com (172.217.22.78) 56(84) bytes of data.
...

$ unshare --user bash
$ ll /usr/bin/ping
-rwsr-xr-x 1 nobody nogroup 72776 Oct  5  2019 /usr/bin/ping*
$ ping google.com
bash: /usr/bin/ping: Operation not permitted
\end{lstlisting}

Um die Sicherheit vor Privilege Escalation durch aus User Namespace Sandboxen zu wahren, ist es wichtig, dass die UID 0 in der Sandbox auf eine unprivilegierte UID außerhalb gemappt wird \cite{lxc-sec, stgraber-unpriv}. Ist sie auf die \texttt{root} UID 0 auf dem Host gemappt spricht man von privilegierten Containern. Diese sind nicht sicher, da ein Sandbox-Ausbruch einer System-Kompromitierung gleichkommt.
Bei unprivilegierten Containern hingegen findet sich ein Angreifer bei einem erfolgreichen Ausbruch als einfacher Nutzer wieder. Im diesem Falle führen User Namespaces keine zusätzliche Unsicherheit ein. Die LXC-Verfasser nenn diesen Modus \textit{root-safe} und gehen sogar so weit, zu sagen, dass es Seccomp, Capabilities und AppArmor/SELinux innerhalb des Containers unnötig macht \cite{lxc-sec}. Die UID/GID Mappings befinden sich in \texttt{/etc/subuid, /etc/subgid}. Sie zeigen auf welchen
Zahlenbereich auf dem Host die UID's im Namespace gemappt werden. Dabei wird eine untere Grenze sowie eine Länge des Bereichs definiert:

\begin{lstlisting}
$ cat /etc/subuid
otter:100000:65536
dachs:165536:65536
\end{lstlisting}

Die UID im User Namespace \texttt{otter} entspricht damit der UID 100000 in dessen Parent Namespace. Die Mappings können über das \texttt{newuidmap} Kommandozeilentool konfiguriert werden. Container Runtimes wie LXC und Docker bieten für diese Einstellung Konfigurationsdateien. Dabei ist es wichtig zu beachten, dass sich die UID Bereiche nicht überlappen. Die Ressourcenlimits sind über \texttt{ulimit} an eine Kernel UID gebunden. Wenn sich zwei Namespaces die selbe Kernel UID teilen,
so kann ein Container zu einem DoS auf einem anderen Service führen \cite{lxc-sec}.

Unprivilegierte Container haben natürlich nicht alle Rechte, die ein privilegierter hätte. Doch in den meisten Fällen braucht \texttt{root} im Container bei Weitem nicht die Berechtigungen typischer \texttt{root} Prozesse. SSH daemon, cron daemon und Netzwerk Mangement finden außerhalb des Containers statt. So kann ein Container in den meisten Fällen mit stark reduzierten Capabilities funktionieren. Docker entfernt standardmäßig zum Beispiel \texttt{mount}, \texttt{mknod}, \texttt{chown},
\texttt{chattr}, sowie das Laden von Kernel Modulen und den Zugriff auf Raw Sockets \cite{docker-sec}.

\begin{lstlisting}[caption={Ein Beispiel Dockerfile für einen NGINX Web Server der nicht als \texttt{root} im Container läuft}]
addgroup --system --gid 101 nginx 
adduser --system --disabled-login --ingroup nginx --no-create-home --home /nonexistent --shell /bin/false --uid 101 nginx
...
USER 101
CMD ["nginx"]
\end{lstlisting}

Die Capabilities können nach einem prinzipiellen Setup innerhalb des Kernels noch weiter reduziert werden, indem wie bereits in Abschnitt \ref{sec:service-acc} beschrieben separate User für Services erstellt werden. Ein Angreifer der einen Service Prozess übernimmt müsste dann zuerst im Container zu \texttt{root} eskalieren.  



\section{Memory Corruption Protection}

Ein weiterer Ansatz zur Härtung, der die Privilege Escalation erschwert sind eine Privilegien-Separierung und Zufälligkeit in der Anordnung des Arbeitsspeichers. PaX wurde 2001 als Patch für den Linux Kernel entwickelt, der versucht Exploits von Memory Corruption Schwachstellen zu verhindern. Zu solchen Schwachstellen zählen uninitialisierter Speicher, unkontrollierter Speicher, Buffer Overflows und Memory Leaks. 
Buffer Overruns ermöglichen es einem Angreifenden, die Return-Addresse des aktuellen Stack-Frames mit einer Addresse zu überschreiben, die zu bösartigem Code führt (Stack Smashing). 

\begin{lstlisting}[language=c,caption={Stack Smashing, das ein Überspringen der \texttt{x = 1} Instruktion bewirkt. Aus \cite{alpeh1996smashing}}]
// ./stack-smash.c
void function(int a, int b, int c) {
   char buffer1[5];
   char buffer2[10];
   int *ret;

   ret = buffer1 + 12;
   (*ret) += 8;
}

void main() {
  int x;

  x = 0;
  function(1,2,3);
  x = 1;
  printf("%d\n",x);
}
\end{lstlisting}


\subsubsection{Least Privilege für Memory Pages}
\label{sec:mem-sec}

PaX löst dieses Problem indem es Teile des Speicher, die durch den Nutzer modifiziert werden können, als nicht ausführbar markiert \cite{shacham_2004}. Speicher ist also entweder schreibbar oder ausführbar ($W \oplus X$). Dies entspricht einer Separierung der Privilegien auf Höhe des Speichers. Dies kann auf Hardware umgesetzt oder auf Software-Ebene mit Performance-Overhead emuliert werden.

Angreifer können Code dadurch nicht mehr injizieren, sondern müssen auf geladenen Library Code (wie \texttt{libc}) zurückgreifen. Bei einem \textit{return-to-libc}-Angriff überschreibt ein Angreifer den Stack mit einer Return-Addresse, die auf eine C Funktion zeigt, und andere Teil des Stacks, die als Argumente intepretiert werden. \texttt{libc} enthält die Wrapper für die Linux System Call API und bietet damit alle Freiheiten zur Interaktion mit dem Kernel. Der $W \oplus X$ Ansatz bietet hiergegen also keine Sicherheit. Damit eine Angreiferin den Kontrollfluss auf eine \texttt{libc}-Funktion lenken kann, muss sie die virtuelle Addresse dieser Funktion kennen.
% TODO return-into-libc attack: solar designer (Alexander Peslyak) https://seclists.org/bugtraq/1997/Aug/63

\subsubsection{Address Space Layout Randomization}
\label{sec:aslr}

Ist der Addressbereich zufällig angeordnet, klappt ein solcher Angriff nur im seltenen Fall, in dem zufällig die richtige Addresse gewählt wird. Dieser Ansatz nennt sich Address Space Layout Randomization (ASLR). PaX randomisiert separat die Addressen dreier relevanter Bereiche: Executable, Mapped und Stack\footnote{Der `Executable' Bereich  enthält den
Code und Daten. Der `Mapped' Bereich enthält den Heap, dynamische Libraries, Thread Stacks und geteilten Speicher.} \cite{pax-aslr}.
Ein Angreifer hat daraufhin nur noch die Möglichkeit, die gewünschte Addresse zu erraten oder die Attacke über viele Versuche zu erzwingen (Brute Force). Ein falscher Versuch versetzt den gekaperten Prozess im Optimalfall in einem Ausnahmezustand (z.B. Crash). Dies sollte als Angriffsindikator registriert werden. Je nachdem wie lange das Wiederherstellen des
angreifbaren Zustandes benötigt und wie groß der Addressraum ist, sind Brute-Force-Angriffe noch praktikabel. Auf 32-bit Architekturen kann PaX den `Mapped'-Bereich nur zu 16 Bit randomisieren. Das bedeutet, das im Durchschnitt nur 32,768 Versuche notwendig sind um die richtige Basis-Addresse dieses Bereichs zu ermitteln. Bei einem \texttt{fork} wird das zufälligte Angeordnete Speicherlayout beibehalten -- im Gegensatz zu einem Context Switch via \texttt{execve}. Speziell Web Server,
die für jeden Request einen neuen Thread starten, haben also für jeden Request dasselbe Address-Layout. Mehrere Tausend Requests auf einem Web Server zu stellen um per Zufall die richtige Addresse zu raten ist eine praktikable und performante Angriffstaktik, die in Experimenten einen Exploit in durschnittlich 3-4 Minuten ermöglicht \cite{shacham_2004}.
Deswegen ist die Verwendung von ASLR nur auf 64-Bit Maschinen verlässlich.
Seit 2005 unterstützt Linux ASLR. Es kann über \texttt{sysctl -w kernel.randomize\_va\_space} kontrolliert werden.

Damit Executables im Speicher arbiträr verschoben werden können müssen sie als positionsunabhängig via \texttt{-pie -fPIE} kompiliert werden. Dies stellte historisch das größte Problem in der Effektivität von ASLR dar -- viele der installierten Anwendungen waren/sind nicht auf diese Weise kompiliert. Seit Debian 9 ist PIE jedoch der Standard für den GCC Compiler \cite{debian-9}.

Kernel Address Space Layout Randomization (KASLR) wendet das beschriebene Prinizip an, um den Kernel gegenüber Exploits zu sichern. Normalerweise ist die Platzierung bestimmter Symbole für eine Kernel-Version konstant und bekannt. In einem simplen Ansatz kann die Basis-Addresse aller Symbole beim Systemstart zufällig platziert werden. Das irrtümliche Zugreifen auf ein nicht-existentes Symbol hat im Falle des Kernels deutliche größere Implikation: ein System Crash. Die Wahrscheinlichkeit
des Erfolgs von Brute Force Attacken sinkt damit in diesem Fall beträchtlich. Dabei ist es wichtig, dass Kernel Addressen nicht über andere Quellen zu erlangen sind \cite{lwn-kaslr}. Mit \texttt{sysctl -w kernel.kptr\_restrict=1}  werden Kernel Pointer, die über \texttt{"\%pK"} ausgegeben werden vorher mit Nullen ersetzt, außer der Nutzer besitzt die nötigen Privilegien\footnote{und hat diese nicht über ein setuid Binary erlangt}. Ebenso sollten die Sichtbarkeit von \texttt{dmesg} Kernel Logs
über \texttt{kernel.dmesg\_restrict} auf Nutzer mit \texttt{CAP\_SYSLOG} Capability beschränkt werden \cite{sysctl-kernel}.

% TODO “ASLR Smack & Laugh Reference” Tilo Müller
% TODO KPTI KAISER
% TODO branch predictor collisions as side-channel

\subsubsection{GCC Optionen}

Der GCC Compiler bietet weitere Optionen, um eine Executable resistent gegenüber Memory Corruption Schwachstellen zu kompilieren \cite{deb-hardening}.

Die \texttt{-fstack-protector} Flag aktiviert `Stack-Smashing-Protection'. 
Diese Sicherung baut auf den \textit{StackGuard} und \textit{ProPolice} Erweiterungen für GCC \cite{stackguard, shacham_2004}.
Durch \textit{StackGuard} wurden Zahlenwerte -- sogenannte `Canaries' -- eingeführt, die als Fallen im Stack vor dem Return-Pointer platziert werden und bei einer unerwarteten Überschreibung einen Buffer-Overflow-Angriff indizieren. Da der Angreifende in Buffer Overflow Attacken nur sequentiell und aufsteigend in den Speicher schreiben kann, ist gesichert, dass er, um die Return Addresse zu überschreiben auch den Canary überschreiben muss. Damit ein Überschreiben der Return Addresse unentdeckt und damit ein Overflow erfolgreich bliebe, müsste der Canary mit demselben Wert überschrieben werden. Um das zu erschweren sollten die Canaries zufällig gewählt werden \cite{stackguard}. 
\textit{ProPolice} erweitert diesen Ansatz, indem es lokalen Variablen und Funktions-Argumente umordnet \cite{shacham_2004}.
Die \texttt{-fstack-protector} Option sichert weniger als etwa 2\% der Funktionen mit der beschriebenen Methode. Die \texttt{-fstack-protector-all} Optionen sichert alle und \texttt{-fstack-protector-strong} versucht eine Balance zwischen beiden zu finden \cite{stack-protector-strong}.
Die Funktion lässt sich nicht nur auf Userspace Code, sondern mittels der \texttt{CONFIG\_CC\_STACKPROTECTOR} Option auch auf den Kernel beziehen \cite{ubuntu-security-features}.

\begin{lstlisting}[language=bash]
$ ./stack-smash
*** stack smashing detected ***: <unknown> terminated
Aborted (core dumped)
\end{lstlisting}

Eine weitere Absicherung gegen Pointer Corrupton kann es sein, Pointer im Speicher zu verschlüsseln und erst beim Laden in ein Register zu entschlüsseln \cite{pointguard}. So kann ein Überrschrieben den Pointer zwar zerstören, aber es kann ohne Kenntniss des Schlüssels kein sinnvoller Wert geschrieben werden. Dies ist nicht in GCC integriert, aber auf Pointer, die in \texttt{glibc} gespeichert sind, angewendet worden \cite{ubuntu-security-features}.

Um ein Programm auf Buffer Overflows zu prüfen, kann des Weiteren das Macro \texttt{\_FORTIFY\_SOURCE} gesetzt werden. Es berechnet die Größe eines Buffers und validiert vor dem Ausführen eines \texttt{memcpy}, \texttt{gets} o.ä., dass die zu kopierenden Daten, nicht die Größe des Buffers übersteigen.
Wird es auf \texttt{1} gesetzt, so finden Checks nur in der Compile Time statt. Auf \texttt{2} finden diese auch während der Laufzeit statt. Erkannte Buffer Overruns führen zum Terminieren des Prozesses \cite{man-fortify}. Diese Option ist sicherer, sollte jedoch von Tests begleitet werden. 

%printf format vulns Wformat-nonliteral (FormatGuard)
%multiple free errors (Conover - Double Free)
%NULL ptr deref (Spengler)
%libsafe

GOT Overwrite Attacks ist eine Methode, bei der ein Angreifer versucht, den Kontrollfluss des Programmes über die Global Offset Table (GOT) zu manipulieren. In dynamisch gelinkten Libraries werden die Addressen für Symbole (z.B. Funktionen) nicht in der Compile-Zeit, sondern erst zur Laufzeit bestimmt. Beim ersten Aufruf einer Methode wird in der `Relocation' durch den Linker die Addresse bestimmt und in die GOT geschrieben. Nun sind standardmäßig die Teile der GOT, in die dabei geschrieben werden schreibbar. 
Verwendet man die `Relocation Read-Only` (RELRO) Flag `ld -z relro` in Kombination mit der BINDNOW Flag `ld -z now`, bestimmt der Linker alle Symbole zur Link-Zeit vor der Laufzeit und markiert danach alle Teile der GOT als read-only. \cite{relro}

Es gibt noch weitere bekannte Memory Corruption Schwachstellen und Mitigationstechniken über GCC Optionen. Diese können über das \texttt{dpkg-buildflags} eingefügt werden. Für unvertrauenswürdige Anwendungen sollte \texttt{DEB\_BUILD\_MAINT\_OPTIONS = hardening=+all} gesetzt werden, um Position Independent Execution und BINDNOW zu aktivieren. Eine Auskunft darüber ob installierte Binaries gehärtet sind bietet \texttt{hardening-check} \cite{debian-hardening}. 
% TODO Diese Optionen können nicht systemweit aktiviert werden, sondern müssen in jedem Build und für jedes Package separat konfiguriert und geprüft werden.


\section{Gehärtete Konfiguration}

Es gibt mehrere weitere Möglichkeiten ein Linux Betriebssystem auf Sicherheit zu konfigurieren.

\subsubsection{Sudo}

\texttt{Sudo} wird benutzt, um Befehle als anderer Nutzer auszuführen und damit potenziell Privilegien zu erhöhen. Dafür wird der User nach dem entsprechenden Passwort gefragt. Das \texttt{/etc/sudoers} File beschreibt dafür welche Nutzer, welche Kommandos von welchen Terminals ausführen dürfen und ob sie dafür nach einem Passwort gefragt werden müssen. Es kann deswegen nur mit erhöhten Privilegien geändert werden. \cite{man-sudo}

Sudo Caching ist ein Mechanismus der die Usability für Sudo-Nutzer erhöhen soll. Der \texttt{timestamp\_timeout} bestimmt, wie viel Zeit in Minuten zwischen zwei \texttt{sudo}-Kommandos verstreichen kann, bevor der Nutzer erneut nach einem Passwort gefragt wird. In der Zwischenzeit cachet sudo die Credentials. Per Default werden die Credentials für 15 Minuten gecachet \cite{man-sudo}.
Dies eröffnet eine potenzielle Schwachstelle. Angreifer können die Zeitstempel des Caches \texttt{/var/db/sudo} beobachten und zum richtigen Zeitpunkt ein Kommando als derselbe Nutzer absetzen, der nun nicht nach einem Passwort gefragt wird. Dies funktioniert nicht, wenn die \texttt{tty\_tickets} Option aktiviert ist. Durch diese Option werden einzelne \texttt{tty} Sessions (Terminals) unter diesem Aspekt als einzigartig betrachtet \cite{attack-sudo-caching}.
Aus den genannten Gründen sollten beide Konfiguration entsprechend gesetzt werden:

\begin{lstlisting}
Defaults        timestamp_timeout=0
Defaults        tty_tickets 
\end{lstlisting}


\subsubsection{Loadable Kernel Modules}

Sogennante Loadbale Kernel Modules (LKM) erweitern den Kernel um Hardware Support (Treiber), um Dateisysteme oder um System Calls. Sie modifizieren also die TCB und sind damit mit äußerster Vorsicht zu beachten. Aus diesem Grund erfordert das Laden neuer Module \texttt{root} Privilegien. Sie werden also nicht zur Privilege Escalation benutzt, helfen jedoch dabei Angriffe zu verstecken oder zu persistieren. Diese Art von Angriff wird auch `Rootkit' genannt. Sie sind schwer zu erkennen
und schwer zu entfernen. Beides wird durch die totale Kontrolle des Rootkits erschwert. Sie verstecken sich selbst, andere Dateien, Prozesse, Netzwerkaktivität und eröffnen authentifizierte Backdoors. \cite{attack-rootkit, attack-lkm}

Auf einem öffentlich zugänglichen System für das die notwendigen Module bekannt sind sollte das Laden von Modulem nach dem Boot deaktiviert werden:

\begin{lstlisting}
$ sudo insmod /lib/modules/5.3.0-51-generic/kernel/crypto/aes_ti.ko
$ sudo rmmod aes_ti

$ sysctl -w kernel.modules_disabled=1
$ sudo insmod /lib/modules/5.3.0-51-generic/kernel/crypto/aes_ti.ko
insmod: ERROR: could not insert module /lib/modules/5.3.0-51-generic/kernel/crypto/aes_ti.ko: Operation not permitted
\end{lstlisting}

Natürlich muss dies davon begleitet werden, dass auch der \texttt{sysctl} System Call z.B. über Seccomp gefiltert wird. Ein Problem ist hierbei, dass nach einer erfolgreichen Privilege Escalation auch diese Filter umgangen werden können. Eine weitere Maßnahme könnte es sein, \texttt{/proc} read-only zu mounten.

\subsubsection{Bash History}

Die Bash Shell zeichnet die Historie der eingegebenen Kommandos auf. Über das \texttt{history} Kommando können die letzten $n$ Kommandos ausgegeben werden. Zum Ende einer Shell Session werden die letzten $n$ Kommandos in die \texttt{\~/.bash\_history}-Datei geschrieben. Über die \texttt{HISTCONTROL}-Variable wird gesteuert welche Kommandos aufgezeichnet werden:

\begin{itemize}
    \item \texttt{ignorespace}: Kommandos, die mit einem Leerzeichen beginnen, werden nicht gespeichert
    \item \texttt{ignoredups}: Duplikate werden nicht gespeichert
    \item \texttt{ignoreboth}: beide werden nicht gespeichert
    \item \texttt{erasedups}: alle Duplikate, die zuvor aufgezeichnet wurden werden gelöscht
    \item leer oder ungesetzt: alle Kommandos werden gespeichert mit Ausnahme von \texttt{HISTIGNORE}
\end{itemize}

Angreifende können diesen Mechanismus benutzen, um nach einer Aktivität ihre `Spuren zu verwischen' und damit einer Verteidigung auszuweichen. Deswegen sollte die \texttt{HISTCONTROL}-Variable auf \texttt{ignoredups} oder ungesetzt sein und Änderungen verhindert werden \cite{attack-histcontrol}. Innerhalb einer Shell Session können Umgebungsvariablen auf \texttt{readonly} gesetzt werden. Dies verhindert Änderungen bis eine neue Shell gestart wird. Diese Maßnahme müsste also von einem
Unterbinden des Startens weiterer Shells verbunden werden. Dies kann eine starke Restriktion für Anwendungen sein, die von \texttt{fork} oder \texttt{exec} abhängen. Eine Alternative könnte es sein, \texttt{setenv} und \texttt{getenv} zu filtern.

Ebenso sollte verhindert werden, dass das \texttt{\~/.bash\_history} File gelöscht wird, z.B.\ indem es per \texttt{chattr +a} als append-only deklariert wird. Die Umgebungsvariablen \texttt{HISTFILE}, \texttt{HISTSIZE}, \texttt{HISTFILESIZE} und \texttt{HISTIGNORE} müssen read-only geschützt sein, damit ein Angreifer weder die History in \texttt{/dev/null} o.ä. umleiten kann, noch die Zahl der aufzuzeichnenden Kommandos auf Null setzen kann, noch alle Kommandos als `zu
ignorieren' definieren kann (z.B. RegEx \texttt{.*}).

Diese Ausführungen addressieren nur die \texttt{bash}-Shell. Nutzen Angreifer andere Shells funktionieren diese Mechanismen natürlich nicht. 
Alternativ können logs über \texttt{rsyslog} an entfernte Systeme weitergeleitet werden, um vor Verlust der Informationen zu sichern \cite{cis}. 
Des Weiteren sollten statt bestimmter Shell Kommandos eher System Events und System Calls über \texttt{auditd} beobachtet werden \cite{cis}. Dabei ist darauf zu achten, dass auch Prozesse beim Starten des Systems (also vor dem Start von \texttt{auditd}) aufgezeichnet werden. Dies lässt sich mit der \texttt{grub} Boot Option \texttt{audit=1} konfigurieren. Die Aufzeichnungen sollten auf sicherheitsrelevanten Systemen nicht in ihrer Größe beschränkt sein. Wenn der definierte
maximale Speicher für die Log-Dateien erreicht wurde, sollte das System bis zur Übertragung halten und keine Logs gelöscht werden:

\begin{lstlisting}
# in /etc/audit/auditd.conf
admin_space_left_action = halt
max_log_file_action = keep_logs
\end{lstlisting}

Ist \texttt{auditd} entsprechend konfiguriert, so lassen sich damit alle möglichen sicherheitsrelevanten Ereignisse aufzeichnen:

\begin{itemize}
    \item Modifikation der Nutzer/Gruppen, z.B. \texttt{/etc/passwd}
    \item Modifikation des MAC, z.B. \texttt{/etc/apparmor}
    \item Modifikation des DAC, z.B. \texttt{chmod, chown}
    \item Unauthorisierte Dateizugriffe
    \item Nutzung privilegierter Kommandos
    \item Laden von Kernel Modulen
    \item u.v.m
\end{itemize}

So kann die Erkennung und die Zurückverfolgung von Angriffen ermöglicht werden und damit zeitig mit den richtig Gegenmaßnahmen vorgegangen werden.


\subsubsection{Partitionierung des Dateisystems}

Eine Partitionierung des physischen Speichers ermöglicht mehrere Sicherungsmaßnahmen.
Einzelne Dateisysteme in separaten Partitionen zu haben sichert die Verfügbarkeit, in dem es Resource Exhaustion (DoS) verhindert. So können Verzeichnisse, die für jeden schreibbar sind\footnote{world-writable} (z.B. \texttt{/tmp}), nicht den Speicher des gesamten Systems verbrauchen, sondern sind durch die Größe ihrer Partition beschränkt.
Des Weiteren ermöglicht es die Sicherung der Vertraulichkeit durch Verschlüsselung einzelner Dateisysteme (wie z.B. Nutzerdateien in \texttt{/home}).
Außerdem lassen sich dadurch Dateisysteme fein-granularer mit unterschiedlichen \texttt{mount}-Optionen konfigurieren. Auch wenn \texttt{/dev} Block Devices enthält, sollte dies in \texttt{/tmp} nicht der Fall sein. Das \texttt{/tmp}-Verzeichnis kann also mit der \texttt{nodev}-Flag gemountet werden. Die \texttt{noexec}-Flag verhindert das Installieren von ausführbaren Programmen und das Erstellen von Hardlinks, die in Kombination mit Setuid-Programmen ausgenützt werden können. Mit
der \texttt{nosuid}-Flag werden Setuid-Programme\footnote{aber auch Linux Capabilities} auf diesem Dateisystem ineffektiv. Einzelne Dateisysteme erfordern sogar überhaupt nicht die Möglichkeit von Nutzern modifiziert zu werden (read-only). \cite{cis}

\begin{table}[H]
\centering
\resizebox{\textwidth}{!}{%
\begin{tabular}{l|ccccc}
Verzeichnis                                 & encryption        & \texttt{noexec}  & \texttt{nodev}    & \texttt{nosuid}   & read-only \\ \hline
\texttt{/tmp}, \texttt{/var/tmp}            & \nocheckb         & \checkb              & \checkb               & \checkb               & \nocheckb        \\
\texttt{/var/log}, \texttt{/var/log/audit}   & \checkb           & \nocheckb               & \nocheckb                & \nocheckb                & \nocheckb        \\
\texttt{/usr}                               & \nocheckb         & \nocheckb               & \nocheckb                & \nocheckb                & \checkb       \\
\texttt{/home}                              & \checkb           & \nocheckb               & \checkb               & \nocheckb                & \nocheckb        \\
\texttt{/dev}                               & \nocheckb         & \checkb              & \nocheckb                & \checkb               & \nocheckb        \\
\texttt{/dev/shm}                           & \nocheckb         & \checkb              & \checkb               & \checkb               & \nocheckb 
\end{tabular}%
}
\end{table}





