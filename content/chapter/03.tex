
\chapter{Zusammenfassung}

Es wurde demonstriert wie Standrad-Linux Mechanismen wie DAC und Capabilties mit Erweiterungen wie Seccomp-Filtern und MAC kombiniert werden können um das Betriebssystem mit minimalen Privilegien zu betreiben. Es wurden Möglichkeiten präsentiert, Virtualisierung in Form von Containern oder virtuellen Maschinen zu nutzen, um Anwendungen abzuschirmen. Des Weiteren wurden Build-Methoden und Konfigurationen dargestellt, mit denen das Betriebbsystem gegen verschiedene Angriffstaktiken
wie das Ausnutzen von Memory-Vulnerabilities oder das Laden von Kernel-Modulen, gesichert werden kann.

Die dargestellten Möglichkeiten zur Systemhärtung sind nicht vollständig. Es wurde unter anderem nicht auf wichtige Punkte wie die Netzwerkkonfiguration, Passwort-Policies, Minimierung der laufenden Services, SSH-Konfiguration, Boot-Mechanismen, Cron-Jobs oder die Handhabung entfernbarer physischer Medien eingegangen.

Linux-Betriebbsysteme sind standardmäßig auf Sicherheit ausgerichtet. Einige der vorgestellten Härtungsmethoden sind zum Beispiel auf Ubuntu mittlerweile per Default eingestellt oder zumindest vorinstalliert und per \texttt{sysctl} konfigurierbar \cite{ubuntu-security-features}. Der Kernel wird von unabhängigen Instanzen auf Sicherheit auditiert und zertifiziert \cite{ubuntu-certifications}. Es gibt ganze Guidelines in Form von Checklisten, nach denen sich die Konfiguration eines Betriebssystems in
ihrer Sicherheit überprüfen und quantifizieren lässt \cite{cis}. Dies lässt sich auch in Form von Regression-Tests automatisieren \cite{qa-regression}. Härtungskonzepte wie das Sandboxing einzelner Anwendungen über Virtualisierung finden darin jedoch keine Erwähnung, da deren Wirksamkeit nicht standardisiert zu prüfen und schwer zu quantifizieren ist. Die dargestellten Mechanismen zeigen jedoch, dass das Kombination aus gehärteter Least-Privilege-Konfiguration mit der Nutzung von
Sandboxes ein geschichtetes Sicherheitskonzept im Sinne der ``Defense in depth'' ermöglichen.
